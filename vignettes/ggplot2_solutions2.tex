%\VignetteIndexEntry{solutions3}
%!Snw weave = knitr
%\VignetteEngine{knitr::knitr}

\documentclass[a4paper,justified,openany]{tufte-handout}\usepackage[]{graphicx}\usepackage[]{color}
%% maxwidth is the original width if it is less than linewidth
%% otherwise use linewidth (to make sure the graphics do not exceed the margin)
\makeatletter
\def\maxwidth{ %
  \ifdim\Gin@nat@width>\linewidth
    \linewidth
  \else
    \Gin@nat@width
  \fi
}
\makeatother

\definecolor{fgcolor}{rgb}{0.102, 0.102, 0.102}
\newcommand{\hlnum}[1]{\textcolor[rgb]{0.2,0.2,0.2}{#1}}%
\newcommand{\hlstr}[1]{\textcolor[rgb]{0.2,0.2,0.2}{#1}}%
\newcommand{\hlcom}[1]{\textcolor[rgb]{0.302,0.302,0.302}{\textit{#1}}}%
\newcommand{\hlopt}[1]{\textcolor[rgb]{0.102,0.102,0.102}{#1}}%
\newcommand{\hlstd}[1]{\textcolor[rgb]{0.102,0.102,0.102}{#1}}%
\newcommand{\hlkwa}[1]{\textcolor[rgb]{0.102,0.102,0.102}{#1}}%
\newcommand{\hlkwb}[1]{\textcolor[rgb]{0.102,0.102,0.102}{#1}}%
\newcommand{\hlkwc}[1]{\textcolor[rgb]{0.2,0.2,0.2}{#1}}%
\newcommand{\hlkwd}[1]{\textcolor[rgb]{0.102,0.102,0.102}{\textbf{#1}}}%
\let\hlipl\hlkwb

\usepackage{framed}
\makeatletter
\newenvironment{kframe}{%
 \def\at@end@of@kframe{}%
 \ifinner\ifhmode%
  \def\at@end@of@kframe{\end{minipage}}%
  \begin{minipage}{\columnwidth}%
 \fi\fi%
 \def\FrameCommand##1{\hskip\@totalleftmargin \hskip-\fboxsep
 \colorbox{shadecolor}{##1}\hskip-\fboxsep
     % There is no \\@totalrightmargin, so:
     \hskip-\linewidth \hskip-\@totalleftmargin \hskip\columnwidth}%
 \MakeFramed {\advance\hsize-\width
   \@totalleftmargin\z@ \linewidth\hsize
   \@setminipage}}%
 {\par\unskip\endMakeFramed%
 \at@end@of@kframe}
\makeatother

\definecolor{shadecolor}{rgb}{.97, .97, .97}
\definecolor{messagecolor}{rgb}{0, 0, 0}
\definecolor{warningcolor}{rgb}{1, 0, 1}
\definecolor{errorcolor}{rgb}{1, 0, 0}
\newenvironment{knitrout}{}{} % an empty environment to be redefined in TeX

\usepackage{alltt}


\usepackage{amsmath}
% Set up the images/graphics package
\usepackage{graphicx}
\setkeys{Gin}{width=\linewidth,totalheight=\textheight,keepaspectratio}
\graphicspath{{graphics/}}

\title{Advanced graphics: practical 3 solutions}
\author[Jumping Rivers]{Jumping Rivers}
\date{}  % if the \date{} command is left out, the current date will be used

% The following package makes prettier tables.  We're all about the bling!
\usepackage{booktabs}

% The units package provides nice, non-stacked fractions and better spacing
% for units.
\usepackage{units}

% The fancyvrb package lets us customize the formatting of verbatim
% environments.  We use a slightly smaller font.
\usepackage{fancyvrb}
\fvset{fontsize=\normalsize}
\newcommand{\cc}{\texttt}
\setcounter{secnumdepth}{2}

\usepackage{microtype}
\IfFileExists{upquote.sty}{\usepackage{upquote}}{}
\begin{document}
\maketitle% this prints the handout title, author, and date

\begin{abstract}
  \noindent This practical aims to guide you through some of the key ideas in
  data manipulation. I've tried to construct this practical in such a way that
  you get to experiment with the various tools. Feel free to experiment!
\end{abstract}


\section{Factors}

When using \cc{ggplot2}, the easiest way of rearranging the graph or to alter labels
is to manipulate the data set. Consider the \cc{mpg} data set:

\begin{knitrout}
\definecolor{shadecolor}{rgb}{1, 1, 1}\color{fgcolor}\begin{kframe}
\begin{alltt}
\hlkwd{data}\hlstd{(mpg,} \hlkwc{package} \hlstd{=} \hlstr{"ggplot2"}\hlstd{)}
\end{alltt}
\end{kframe}
\end{knitrout}

Suppose we generate a scatter plot of engine displacement against highway mpg.

\begin{knitrout}
\definecolor{shadecolor}{rgb}{1, 1, 1}\color{fgcolor}\begin{kframe}
\begin{alltt}
\hlstd{g} \hlkwb{=} \hlkwd{ggplot}\hlstd{(}\hlkwc{data}\hlstd{=mpg,} \hlkwd{aes}\hlstd{(}\hlkwc{x}\hlstd{=displ,} \hlkwc{y}\hlstd{=hwy))} \hlopt{+}
  \hlkwd{geom_point}\hlstd{()}
\end{alltt}
\end{kframe}
\end{knitrout}

\noindent Next, we add a loess line, conditional on the drive type:

\begin{knitrout}
\definecolor{shadecolor}{rgb}{1, 1, 1}\color{fgcolor}\begin{kframe}
\begin{alltt}
\hlstd{g} \hlopt{+} \hlkwd{stat_smooth}\hlstd{(}\hlkwd{aes}\hlstd{(}\hlkwc{colour}\hlstd{=drv))}
\end{alltt}
\end{kframe}
\end{knitrout}

\noindent While this graph is suitable for exploring the data; for publication,
we would like to rename the axis and legend labels. To change the axis labels,
we can rename the data frame columns or use \cc{xlab} and \cc{ylab}. To change
the order of the legend, we need to manipulate the data. Since \cc{drv} is
a character, we could use:

\begin{knitrout}
\definecolor{shadecolor}{rgb}{1, 1, 1}\color{fgcolor}\begin{kframe}
\begin{alltt}
\hlstd{mpg[mpg}\hlopt{$}\hlstd{drv} \hlopt{==} \hlstr{"4"}\hlstd{,]}\hlopt{$}\hlstd{drv} \hlkwb{=} \hlstr{"4wd"}
\hlstd{mpg[mpg}\hlopt{$}\hlstd{drv} \hlopt{==} \hlstr{"f"}\hlstd{,]}\hlopt{$}\hlstd{drv} \hlkwb{=} \hlstr{"Front"}
\hlstd{mpg[mpg}\hlopt{$}\hlstd{drv} \hlopt{==} \hlstr{"r"}\hlstd{,]}\hlopt{$}\hlstd{drv} \hlkwb{=} \hlstr{"Rear"}
\end{alltt}
\end{kframe}
\end{knitrout}

\noindent However the legend will still be ordered alphabetically.
Instead, we can use a factor:

\begin{knitrout}
\definecolor{shadecolor}{rgb}{1, 1, 1}\color{fgcolor}\begin{kframe}
\begin{alltt}
\hlcom{##Reload the data just to make sure}
\hlkwd{data}\hlstd{(mpg,} \hlkwc{package}\hlstd{=}\hlstr{"ggplot2"}\hlstd{)}
\hlstd{mpg}\hlopt{$}\hlstd{drv} \hlkwb{=} \hlkwd{factor}\hlstd{(mpg}\hlopt{$}\hlstd{drv,} \hlkwc{labels} \hlstd{=} \hlkwd{c}\hlstd{(}\hlstr{"4wd"}\hlstd{,} \hlstr{"Front"}\hlstd{,} \hlstr{"Rear"}\hlstd{))}
\end{alltt}
\end{kframe}
\end{knitrout}
\noindent To change the order of the, we need to use the \cc{factor} function:
\begin{knitrout}
\definecolor{shadecolor}{rgb}{1, 1, 1}\color{fgcolor}\begin{kframe}
\begin{alltt}
\hlstd{mpg}\hlopt{$}\hlstd{drv} \hlkwb{=} \hlkwd{factor}\hlstd{(mpg}\hlopt{$}\hlstd{drv,}
                 \hlkwc{levels} \hlstd{=} \hlkwd{c}\hlstd{(}\hlstr{"Front"}\hlstd{,} \hlstr{"Rear"}\hlstd{,} \hlstr{"4wd"}\hlstd{))}
\end{alltt}
\end{kframe}
\end{knitrout}

\noindent The legend now displays the labels in the order: \cc{Front}, \cc{Rear} and \cc{4wd}.

\begin{figure}[t]
\centering
\begin{knitrout}
\definecolor{shadecolor}{rgb}{1, 1, 1}\color{fgcolor}

{\centering \includegraphics[width=\maxwidth]{knitr_figure/graphics-F1-1} 

}



\end{knitrout}
\caption{Final figure from section 2.}\label{F1}
\end{figure}

\section{Aphids}

This data set consists of seven observations on cotton aphid counts on twenty
randomly chosen leaves in each plot, for twenty-seven treatment-block
combinations. The data were recorded in July 2004 in Lamesa, Texas. The
treatments consisted of three nitrogen levels (blanket, variable and none),
three irrigation levels (low, medium and high) and three blocks, each being a
distinct area. Irrigation treatments were randomly assigned within each block as
whole plots. Nitrogen treatments were randomly assigned within each whole block
as split plots.

\begin{knitrout}
\definecolor{shadecolor}{rgb}{1, 1, 1}\color{fgcolor}\begin{kframe}
\begin{alltt}
\hlkwd{data}\hlstd{(aphids,} \hlkwc{package}\hlstd{=}\hlstr{"jrSouth"}\hlstd{)}
\end{alltt}
\end{kframe}
\end{knitrout}

\noindent The sampling times are once per week.

\newthought{Reproduce} figure \ref{F1}. Here are some hints to get you started.
The key idea is to think of the plot in terms of layers. So
\begin{itemize}
\item Leave the ordering of factors to the end
\item The plot contains a combination of \cc{geom\_line} and \cc{geom\_point}.
\item You can change the x-axis label using
\begin{knitrout}
\definecolor{shadecolor}{rgb}{1, 1, 1}\color{fgcolor}\begin{kframe}
\begin{alltt}
 \hlopt{+} \hlkwd{xlab}\hlstd{(}\hlstr{"Time"}\hlstd{)}
\end{alltt}
\end{kframe}
\end{knitrout}
\item Change the theme using \cc{theme\_bw()}
\end{itemize}
\newpage
\begin{knitrout}
\definecolor{shadecolor}{rgb}{1, 1, 1}\color{fgcolor}\begin{kframe}
\begin{alltt}
\hlcom{##Code for figure 1}
\hlstd{aphids}\hlopt{$}\hlstd{Block} \hlkwb{=} \hlkwd{factor}\hlstd{(aphids}\hlopt{$}\hlstd{Block)}
\hlstd{aphids}\hlopt{$}\hlstd{Water} \hlkwb{=} \hlkwd{factor}\hlstd{(aphids}\hlopt{$}\hlstd{Water,}
                      \hlkwc{levels}\hlstd{=}\hlkwd{c}\hlstd{(}\hlstr{"Low"}\hlstd{,} \hlstr{"Medium"}\hlstd{,} \hlstr{"High"}\hlstd{))}
\hlstd{ga} \hlkwb{=} \hlkwd{ggplot}\hlstd{(}\hlkwc{data}\hlstd{=aphids)} \hlopt{+}
  \hlkwd{geom_point}\hlstd{(}\hlkwd{aes}\hlstd{(Time, Aphids,} \hlkwc{colour}\hlstd{=Block))} \hlopt{+}
  \hlkwd{facet_grid}\hlstd{(Nitrogen} \hlopt{~} \hlstd{Water)} \hlopt{+}
  \hlkwd{geom_line}\hlstd{(}\hlkwd{aes}\hlstd{(Time, Aphids,} \hlkwc{colour}\hlstd{=Block))} \hlopt{+}
  \hlkwd{theme_bw}\hlstd{()}

\hlkwd{print}\hlstd{(ga)}
\end{alltt}
\end{kframe}
\end{knitrout}



\section*{Solutions}

Solutions are contained within this package:
\begin{knitrout}
\definecolor{shadecolor}{rgb}{1, 1, 1}\color{fgcolor}\begin{kframe}
\begin{alltt}
\hlkwd{library}\hlstd{(jrGgplot2)}
\hlkwd{vignette}\hlstd{(}\hlstr{"ggplot2_solutions2"}\hlstd{,} \hlkwc{package}\hlstd{=}\hlstr{"jrSouth"}\hlstd{)}
\end{alltt}
\end{kframe}
\end{knitrout}
\end{document}
